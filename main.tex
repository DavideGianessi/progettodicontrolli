\documentclass[a4paper, 11pt]{article}
\usepackage[margin=3cm]{geometry}
\usepackage[]{fontenc}
\usepackage[utf8]{inputenc}
\usepackage[italian]{babel}
\usepackage{geometry}
\usepackage{amsmath}
\usepackage{amssymb}
\usepackage{gensymb}
\usepackage{graphicx}
\usepackage{psfrag,amsmath,amsfonts,verbatim}
\usepackage{xcolor}
\usepackage{color,soul}
\usepackage{fancyhdr}
\usepackage{indentfirst}
\usepackage{graphicx}
\usepackage{newlfont}
\usepackage{amssymb}
\usepackage{amsmath}
\usepackage{latexsym}
\usepackage{amsthm}
\usepackage{subfigure}
\usepackage{subcaption}
\usepackage{psfrag}
\usepackage{footnote}
\usepackage{graphics}
\usepackage{color}
\usepackage{hyperref}
\usepackage{tikz}
\usepackage{float}

\usetikzlibrary{snakes}
\usetikzlibrary{positioning}
\usetikzlibrary{shapes,arrows}

	
	\tikzstyle{block} = [draw, fill=white, rectangle, 
	minimum height=3em, minimum width=6em]
	\tikzstyle{sum} = [draw, fill=white, circle, node distance=1cm]
	\tikzstyle{input} = [coordinate]
	\tikzstyle{output} = [coordinate]
	\tikzstyle{pinstyle} = [pin edge={to-,thin,black}]

\newcommand{\courseacronym}{CAT}
\newcommand{\coursename}{Controlli Automatici - T}
\newcommand{\tipology}{B}
\newcommand{\trace}{1}
\newcommand{\projectname}{Controllo di una tavola rotante motorizzata}
\newcommand{\group}{49}

%opening
\title{ \vspace{-1in}
		\huge \strut \coursename \strut 
		\\
		\Large  \strut Progetto Tipologia \tipology - Traccia \trace 
		\\
		\Large  \strut \projectname\strut
		\\
		\Large  \strut Gruppo \group\strut
		\vspace{-0.4cm}
}
\author{ Isacco Briali, Emanuele Coacci, Davide Gianessi}
\date{}

\begin{document}

\maketitle
\vspace{-0.5cm}
In questo progetto studiamo il controllo di una tavola rotante motorizzata, collegata ad un motore tramite un giunto cardanico. La dinamica del sistema è descritta dalle seguenti equazioni differenziali:
%
\begin{subequations}\label{eq:system}
\begin{align}
	    J\dot{w} &= \tau(\theta)C_m - \beta\omega - k\theta,
        \\
        dove \hspace{0,2cm} \spac \tau(\theta) &= \frac{\cos{\alpha}}{1-(\sin{\alpha\cos{\theta)^2}}}
\end{align}
\end{subequations}
%

Assumiamo $\theta$(t) la posizione angolare della tavola e $\omega$(t) la sua velocità angolare. Otteniamo dal sistema scritto in precedenza che $\tau$($\theta$) è il rapporto di trasmissione del giunto cardanico funzione $\theta$ e dell'angolo tra i due alberi $\alpha$, mentre J è il momento di inerzia della tavola. Completando le equazioni abbiamo $C_m$ come imput di controllo, ossia la coppia generata dal motore elettrico, e i due coeficienti $\beta$ e k che sono rispettivamente l'attrito viscoso e l'elasticità del disco. Supponiamo in oltre che la posizione angolare $\theta$ della tavola possa essere misurata.

Per la realizzazzione di questo progetto, per maggiore comodità, usabilità ed efficienza, abbiamo adoperato due diversi ambienti: Python e MATLAB. Con il primo abbiamo effettuato tutti i calcoli che verranno riportati in seguito nella relazione. Il secondo, invece, lo abbiamo utilizzato per effettuare i test del regolatore sul sistema lineare e sul sistema non lineare.

\section{Espressione del sistema in forma di stato e calcolo del sistema linearizzato intorno ad una coppia di equilibrio}

Innanzitutto, esprimiamo il sistema~\eqref{eq:system} nella seguente forma di stato
%
\begin{subequations}
\begin{align}\label{eq:state_form}
	\dot{x} &= f(x,u)
	\\
	y &= h(x,u).
\end{align}
\end{subequations}
%
Pertanto, andiamo a individuare lo stato $x$, l'ingresso $u$ e l'uscita $y$ del sistema come segue 
%
\begin{align*}
x := \begin{bmatrix}
	    x_1\\
            x_2     
	\end{bmatrix} := \begin{bmatrix}
	    \theta\\
        \omega     
	\end{bmatrix}, \quad u := C_m, \quad y := \theta.
\end{align*}
%
Coerentemente con questa scelta, ricaviamo dal sistema~\eqref{eq:system} la seguente espressione per le funzioni $f$ ed $h$
%
\begin{align*}
	f(x,u) &:= \begin{bmatrix}
	    f_1(x,u)\\
            f_2(x,u)     
	\end{bmatrix} := \begin{bmatrix}
	    x2\\
            \frac{\frac{\cos{\alpha}}{1-(\sin{\alpha}\cos{x_1})^2}C_m - \beta x_2 - kx_1}{J}     
	\end{bmatrix}
	\\
	h(x,u) &:= x_1.
\end{align*}
%
Una volta calcolate $f$ ed $h$ esprimiamo~\eqref{eq:system} nella seguente forma di stato
%
\begin{subequations}\label{eq:our_system_state_form}
\begin{align}
	\begin{bmatrix}
		\dot{x}_1
		\\
		\dot{x}_2
	\end{bmatrix} &= \begin{bmatrix}
	    x2\\
            \frac{\frac{\cos{\alpha}}{1-(\sin{\alpha}\cos{x_1})^2}C_m - \beta x_2 - kx_1}{J}     \end{bmatrix} \label{eq:state_form_1}
	\\
	y &= x_1.
\end{align}
\end{subequations}
%
Da specifiche ci viene dato il valore di equilibrio $ x_e = \begin{bmatrix}
		x_{1,e}
		\\
		x_{2,e}
	\end{bmatrix} = 
    \begin{bmatrix}
		\theta_e
		\\
		\omega_e
	\end{bmatrix} =
 \begin{bmatrix}
		120^\circ
		\\
		0
	\end{bmatrix}$

A partire dal quale troviamo la coppia di equilibrio $(x_e, u_e)$ di~\eqref{eq:our_system_state_form}, risolvendo il seguente sistema di equazioni
%
\begin{align}
	\begin{bmatrix}
		\dot{x}_1
		\\
		\dot{x}_2
	\end{bmatrix} = \begin{bmatrix}
		0
		\\
		0
	\end{bmatrix} = \begin{bmatrix}
		0
		\\
		\frac{\sqrt{3}u-125\pi}{1500}
	\end{bmatrix}
\end{align}
%
dal quale otteniamo
%
\begin{align}
	x_e := \begin{bmatrix}
		120^\circ
		\\
		0
	\end{bmatrix},  \quad u_e = 226.7249\label{eq:equilibirum_pair}
\end{align}
%
Definiamo le variabili alle variazioni $\delta x$, $\delta u$ e $\delta y$ come 
%
\begin{align*}
	\delta x &= x-x_e, 
	\quad
	\delta u = u-u_e, 
	\quad
	\delta y = y-y_e.
\end{align*}
%
L'evoluzione del sistema espressa nelle variabili alle variazioni pu\`o essere approssimativamente descritta mediante il seguente sistema lineare
%
\begin{subequations}\label{eq:linearized_system}
\begin{align}
	\delta \dot{x} &= A\delta x + B\delta u
	\\
	\delta y &= C\delta x + D\delta u,
\end{align}
\end{subequations}
%
dove le matrici $A$, $B$, $C$ e $D$ vengono calcolate come
%
\begin{subequations}\label{eq:matrices}
\begin{align}
	A &= \begin{bmatrix}
		\frac{\partial f_1(x,u)}{\partial x_1} & \frac{\partial f_1(x,u)}{\partial x_2}
		\\
		\frac{\partial f_2(x,u)}{\partial x_1} & \frac{\partial f_2(x,u)}{\partial x_2}
	\end{bmatrix}_{\substack{x = x_e\\u = u_e}} = \begin{bmatrix}
		  0 & 1
		\\
		-0.06454 & -0.00075
	\end{bmatrix}
	\\
	B &= \begin{bmatrix}
		\frac{\partial f_1(x,u)}{\partial u}
		\\
		\frac{\partial f_2(x,u)}{\partial u}
	\end{bmatrix}_{\substack{x = x_e\\u = u_e}} = \begin{bmatrix}
	       0 \\ 0.001154
	\end{bmatrix}
	\\
	C &= \begin{bmatrix}
		\frac{\partial h(x,u)}{\partial x_1}
		&
		\frac{\partial h(x,u)}{\partial x_2}
	\end{bmatrix}_{\substack{x = x_e\\u = u_e}} = \begin{bmatrix}
	    1 & 0
	\end{bmatrix}
	\\
	D &= \begin{bmatrix}
		\frac{\partial h(x,u)}{\partial u}
	\end{bmatrix}_{\substack{x = x_e\\u = u_e}} = \begin{bmatrix}
	    0
	\end{bmatrix}
\end{align}
\end{subequations}
%
\section{Calcolo Funzione di Trasferimento}

In questa sezione, andiamo a calcolare la funzione di trasferimento $G(s)$ dall'ingresso $\delta u$ all'uscita $\delta y$ mediante la seguente formula 
%
%
\begin{align}\label{eq:transfer_function}
G(s) = C(sI -A)^{-1} B+D = \frac{0.0011547}{s^2+0.00075s+0.18546}.
\end{align}

Dunque il sistema linearizzato~\eqref{eq:linearized_system} è caratterizzato dalla funzione di trasferimento~\eqref{eq:transfer_function} con $2$ poli complessi coniugati $p_1p_2 = -0.000375 \pm 0.254j $ e uno\\ In Figura ~\ref{Figura1} mostriamo il corrispondente diagramma di Bode. 
\begin{figure}[H]
    \centering
\includegraphics[width=110mm]{}
    \caption{}
    \label{Figura1}
\end{figure}

\section{Mappatura specifiche del regolatore}
\label{sec:specifications}

Le specifiche da soddisfare sono
\begin{itemize}
	\item[1)] Errore a regime $\space |e_{\infty}| \le e^* = 0.005 $ con riferimeto a gradino
	\\
	\item[2)] $M_f \ge 30\degree$ 
 \\
	\item[3)] $S\%\ge 2\%$: ovvero $M_f \ge 77.97\degree$
 \\
	\item[4)] $T_{a,5} = 0.03s$. 
 \\
	\item[5)]$d(t)$ attenuato di almeno $40dB$ in $\begin{bmatrix}
	    0 , & 0.5
	\end{bmatrix}$
 \\
	\item[6)]$n(t)$ attenuato di almeno $63dB$ in $\begin{bmatrix}
	    10^5 , & 10^6
	\end{bmatrix}$
\end{itemize}
%
Andiamo ad effettuare la mappatura punto per punto le specifiche richieste.

\begin{itemize}
	\item[1)] L'errore a reggime $e_{\infty}$ $ < $ 0.005 in risposta a un gradino w(t) = 1(t) e d(t) = 1 (t) .\\
	\item[2)] Le specifiche 2 e 3 sono entrambe sul margine di fase, perci\`o andiamo a soddisfare la pi\`u stringente. Otteniamo quindi $M_f\ge77.97\degree$.
	\\
	\item[3)] Dalla specifica 4, sul tempo di assestamento $T_{a,5}$, ricaviamo: $\omega _c \ge \frac{300}{0,03 M_f} \approx 128.25$rad/s.
	\item[4)] Dalla specifica 5 ricaviamo $|L(j\omega)|_{dB} \ge 40dB$ in $\begin{bmatrix}
	    0 , & 0.5
	\end{bmatrix}$.\\
	\item[5)] Dalla specifica 6 ricaviamo $|L(j\omega)|_{dB} \le -63dB$ in$\begin{bmatrix}
	    10^5 , & 10^6
	\end{bmatrix}$.\\
\end{itemize}

Pertanto, in Figura ~\ref{Figura2}, mostriamo il diagramma di Bode della funzione di trasferimento $G(s)$ con le zone proibite emerse dalla mappatura delle specifiche.
\begin{figure}[H]
    \centering
\includegraphics[width=110mm]{images/figure_3.jpg}
    \caption{}
    \label{Figura2}
\end{figure}
\section{Sintesi del regolatore statico}
\label{sec:static_regulator}

In questa sezione progettiamo il regolatore statico $R_s(s)$ partendo dalle analisi fatte in sezione~\ref{sec:specifications}.
Per ridurre l'errore a regime del sistema aumentiamo il guadagno statico $\mu$ del regolatore in modo tale da portare $e_\infty \space $ sotto 0.005.

Dunque, definiamo la funzione estesa $G_e(s) = R_s(s)G(s)$ e, in Figura ~\ref{Figura3}, mostriamo il suo diagramma di Bode per verificare se e quali zone proibite vengono attraversate.
\begin{figure}[H]
    \centering
\includegraphics[width=110mm]{images/figure_4.jpg}
    \caption{}
    \label{Figura3}
\end{figure}
Da Figura ~\ref{Figura3}, emerge che la $G_e(s)$ non rispetta i vincoli sul margine di fase $M_f$ e sul tempo di assestamento $\omega_{c,min}$


\section{Sintesi del regolatore dinamico}

In questa sezione, progettiamo il regolatore dinamico $R_d(s)$. 
%
Dalle analisi fatte in Sezione~\ref{sec:static_regulator}, notiamo di essere nello Scenario di tipo B. Dunque, progettiamo $R_d(s)$ riccorrendo ad una rete anticipatrice. Utilizziamo le formule di inversione per calcolare $\alpha$ e $\tau$ di $R_d(s)$, prendendo $M_f^*=M_{f,spec}+10$ e $\omega_c=256.51$. Otteniamo quindi:

\begin{align}\label{eq:R_s}
R_s(s)=\frac{1+\tau s}{1+\tau\alpha s}
\end{align}

In Figura ~\ref{Figura4}, mostriamo il diagramma di Bode della funzione d'anello $L(s) = R_d(s) G_e(s)$
\begin{figure}[H]
    \centering
\includegraphics[width=110mm]{images/figure_5.jpg}
    \caption{}
    \label{Figura4}
\end{figure}
\section{Test sul sistema linearizzato}

In questa sezione, testiamo l'efficacia del controllore progettato sul sistema linearizzato considerando in primo luogo le risposte al riferimento a gradino w(t) e ai rumori $n(t)$ e $d(t)$, per poi ottenere l'uscita complessiva $y_{\text{tot}}$ sommando le singole risposte per il princio di sovrapposizione degli effetti. Specifichiamo inoltre che, come detto all'inizio, il test è stato effettuato su MATLAB; tuttavia, il grafico che verrà riportato in seguito è stato realizzato con Python.

%
\begin{subequations}\label{eq:system}
\begin{align}
	    y_w &= F(s)W(s)
        \\
        y_d &= S(s)D(s)
        \\
        y_n &= -F(s)N(s)
        \\
        y_{\text{tot}}&= y_w+y_d+y_n
\end{align}
\end{subequations}
%
\begin{figure}[H]
    \centering
\includegraphics[width=110mm]{images/figure_6.jpg}
    \caption{}
    \label{Figura6}
\end{figure}


Da Figura 5 notiamo che l'uscita soddisfa i vincoli relativi alla sovraelongazione percentuale, al tempo di assestamento e all'errore a regime.

\section{Test sul sistema non lineare}

In questa sezione, testiamo l'efficacia del controllore progettato sul modello non lineare.

Per fare ciò dovremo:

\begin{itemize}
	\item[1)] Aggiungere $u_e$ in ingresso al sistema come parte dell'ingresso totale $u$, quindi $u = \delta u + u_e$. L'uscita sarà $y$ anziché $\delta y$.
 \\
        \item[2)]Sottrarre $y_e$ in uscita al sistema, così da avere l'uscita $\delta y = y - y_e$. Questo consentirà di calcolare l'errore $e$ sottraendo $\delta y$ dal riferimento w e utilizzarlo come ingresso per il regolatore.
\end{itemize}

In figura mostreremo l'uscita del sistema non lineare considerando i $n(t)$ e $d(t)$.
\begin{figure}[H]
    \centering
\includegraphics[width=110mm]{images/figure_10.jpg}
    \caption{}
    \label{Figura9}
\end{figure}
\section{Sistema lineare e non lineare, conclusioni}

In conclusione, possiamo apprezzare come il sistema lineare in anello chiuso rispetti le specifiche assegnate e, una volta esaurito il transitorio, l'uscita tenda a seguire il riferimento w(t) assegnato. Il sistema non lineare, invece, mostra che il regolatore si comporta diversamente rispetto alle specifiche. Questo accade perché il sistema non è predisposto alla linearizzazione o, comunque, presenta un'elevata imprecisione rispetto al sistema originale, che causa tale comportamento indesiderato.

\section{Punti opzionali: la dinamica del sistema}



\section{Punti opzionali: }

Supponendo un riferimento w(t) ≡ 0, esplorare il range di condizioni iniziali dello stato del sistema non lineare (nell’intorno del punto di equilibrio) tali per cui l’uscita del sistema in anello chiuso converga a h(xe,ue).





\end{document}
